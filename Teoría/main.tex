\documentclass{article}


% Idioma: español
\usepackage[spanish,es-lcroman]{babel} % Características del idioma
\usepackage[utf8]{inputenc} % Acentos

% Aspecto de página
\usepackage[a4paper]{geometry} % Márgenes
\usepackage{fancyhdr} % Headers
\usepackage{parskip} % Espacio entre párrafos
\usepackage{microtype} % Kerning y mejoras tipográficas

% Columnas y tablas
\usepackage{multicol}

% Imágenes
\usepackage{graphicx}

% Símbolos
\usepackage{amsmath} % \text, \overset, \underset
\usepackage{xcolor}
\usepackage{soul} % \so, \st, \hl
\usepackage{cancel} % \cancel, \bcancel, \xcancel, \cancelto

% Listas
\usepackage{enumerate} % Numeración de listas

% Dibujos y gráficas
\usepackage{tikz} 
\usepackage{pgfplots}
\usetikzlibrary{shapes,arrows,decorations.markings,shapes.misc,shapes.geometric,calc,pgfplots.groupplots}
\pgfplotsset{compat=1.13}

% Links
\usepackage{hyperref}
\hypersetup{colorlinks}

% Otros
\usepackage[spanish,colorinlistoftodos,obeyDraft]{todonotes} % ToDo
\usepackage{pdfpages} % Incluir PDFs
\usepackage[acronym]{glossaries} % Siglas y glosario
\usepackage{acronym} % Acrónimos

% Código
\usepackage{listingsutf8}
\lstset{
	inputencoding=utf8,
	extendedchars=true,
	literate=%
	{á}{{\'a}}1
	{é}{{\'e}}1
	{í}{{\'i}}1
	{ó}{{\'o}}1
	{ú}{{\'u}}1
	{Á}{{\'A}}1
	{É}{{\'E}}1
	{Í}{{\'I}}1
	{Ó}{{\'O}}1
	{Ú}{{\'U}}1
	{ñ}{{\~n}}1
	{Ñ}{{\~N}}1
	{¿}{{>}}1
	{¡}{{<}}1
}

\usepackage{textcomp}
\lstset{
	language=C++,
	directivestyle={\color{black}},
	%
	emph={int,char,double,float,unsigned,void,bool,ll,vi, vvi,list,queue,mli,mlc,unordered_set,unordered_map,set,vector,long},
	emphstyle={\color{blue}},
	%
	keywordstyle={\color{black}\sffamily\bfseries},
	%morekeywords={push,front,pop,push_back,pop_back,push_front,pop_front,top,empty,find,end,begin,cin,cout,erase},
	%
	basicstyle=\tt,
	commentstyle=\itshape\sffamily\color{green!50!black},
	identifierstyle=\color{black},
	stringstyle={\color{blue}},
	showstringspaces=false,
	columns=flexible,
	%
	numbers=none,
	numberstyle=\color{gray},
	firstnumber = 1,
	stepnumber=2,
	tabsize =2,
	escapechar=¬,
	breaklines=true,
	upquote=true
}

\title{}
\author{}
\date{}

\begin{document}
	
%	\maketitle
%	
%	\begin{abstract}\end{abstract}
%	
%	\tableofcontents
	
	\section{Versión 1}
	La primera versión consiste en realizar una búsqueda en anchura estándar sobre el árbol de soluciones. Es decir, se recorre cada posible solución, pero recorriendo todas las posibilidades de longitud $n$ antes de pasar a las soluciones de longitud $n+1$. Así nos aseguramos de que pasamos por la solución óptima antes que cualquier otra.
	
	El código del que partimos es el siguiente:
	
	\lstinputlisting{v1.cpp}
	
	\subsection{Restricciones}
	\begin{enumerate}
		\item Está en C++.
		\item A partir de 1000 tarda demasiado.
		\item No se tienen en cuenta operaciones encestadas: (17-2)/(2+3)
		\item Tiene más paréntesis de la cuenta.
	\end{enumerate}

	\subsection{A hacer}
	\begin{enumerate}
		\item Ver si ayuda optimizar factorizando.
		\item Ver cuál sería la estructura óptima (x*y+-z).
	\end{enumerate}

\end{document}
